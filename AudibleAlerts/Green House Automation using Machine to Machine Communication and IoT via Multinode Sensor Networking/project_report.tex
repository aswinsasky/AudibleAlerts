\documentclass[12pt,a4paper]{report}
\usepackage{graphicx}
\usepackage{amsmath}
\usepackage{fancyhdr}
\usepackage{cite}
\usepackage{framed}
\usepackage{a4wide}
\usepackage{float}
\usepackage{epsfig}
\usepackage{longtable}
\usepackage{enumerate}
\usepackage{afterpage}
\usepackage{multirow}
\usepackage{ragged2e}
\usepackage{gensymb}
\usepackage{amsfonts} 
\usepackage[left=3.5cm,top=1.5cm,right=3cm,bottom=4cm]{geometry}
\usepackage{setspace}           
\usepackage{float}
\usepackage{txfonts}
\usepackage{lipsum}

\newcommand{\Usefont}[1]{\fontfamily{#1}\selectfont}

\usepackage{lscape} % for landscape tables
\renewcommand{\baselinestretch}{1.7} 

\usepackage{blindtext}
\usepackage{xpatch}
\usepackage{url}
\usepackage{leqno}
\usepackage{subcaption}

\linespread{1.5}
\usepackage[intoc, english]{nomencl}
\hyphenpenalty=5000
\tolerance=1000
\usepackage[nottoc]{tocbibind}
\bibliographystyle{IEEEtran}
\renewcommand{\bibname}{References}

%*******************************************************************
%                        Header and Footer   
% This is not required in Technical reports submitted to CET ECE department.
% Please leave it commented                       
%*******************************************************************
%\pagestyle{fancy}
%\fancyhead{}
%\header and footer section
%\renewcommand\headrulewidth{0.1pt}
%\fancyhead[L]{\footnotesize \leftmark}
%\fancyhead[R]{\footnotesize \thepage}
%\renewcommand\headrulewidth{0pt}
%\fancyfoot[R]{\small College of Engineering Trivandrum}
%\renewcommand\footrulewidth{0.1pt}
%\fancyfoot[C]{2020 - 2021}
%\fancyfoot[L]{\small Title of the Seminar/Project}
%*******************************************************************


%*********************Figures*****************************
% Save all figures in the folder figures and include them in your 
% report using the command \includegraphics{figure-name}

\graphicspath{{figures/}}

% figure files can be in jpeg,jpg, png or pdf formats
%*******************************************************************


\begin{document}
	
	
%****The entries in this section are to be filled in by the student with appropriate values *************

% These values are used thoroughout the report 
% please fill in the appropriate values in the brackets {}

\gdef \title{ Audible Alert} % Project title
%\gdef \author{Student Name}	 %student name
\gdef \dept{Computer Science and Engineering} %Department
\gdef \degree{Bachelor of Technology} %degree
\gdef \branch{Computer Science and Engineering} %branch
\gdef \college{LBS College of Engineering} % Name of the College
\gdef \collegeplace{Kasaragod} % Location of the College
\gdef \studentA{Akhil Joseph} %Project batch member 1
\gdef \studentAroll{KSD21CS012} % Project batch member 1 ktu id
\gdef \studentB{Aravind Ramesh} %Project batch member 2
\gdef \studentBroll{KSD21CSO23} % Project batch member 2 ktu id

\gdef \studentC{Aswin k} %Project batch member 3
\gdef \studentCroll{KSD21CS028} % Project batch member 3 ktu id

\gdef \studentD {Muhammed Irshad k} %Project batch member 4
\gdef \studentDroll{KSD21CS072} % Project batch member 4 ktu id

\gdef \guide{Prof. Rahul C } %Project guide
\gdef \guidedes{Assistant Professor}%project guide designation

\gdef \guideco{Prof. Project coguide} %Project coguide
\gdef \guidecodes{Assistant Professor}%project coguide designation

\gdef \guideext{Prof. Project ext guide} %Project external organisation guide
\gdef \guideextdes{Engineer/Scientist}%project external guide designation
\gdef \guideextorg{External guide organization} % Project external guide organization

\gdef \projcordinatorA{Prof. Anver S R}% Project coordinator 1 
\gdef \projcordinatorAdes{Assistant Professor}% Project coordinator 1 designation

\gdef \projcordinatorB{Prof. Project coordinator 2} % Project coordinator 2 
\gdef \projcordinatorBdes{Assistant Professor}% Seminar coordinator 2 designation

\gdef \hod{Prof . Anver S R} %Head of Department
\gdef \hoddes{Professor and Head} %HOD designation

\gdef \acadyear{2022 - 23} % Academic year
\gdef \month{November 2022} %Month of Report submission
\gdef \date{30-11-2022} %Date of signing the declaration

%*******************************************************************
% The font pages. The source tex files are there in the folder
\newenvironment{coverpage}
\section{\thispagestyle{empty}}
\begin{titlepage}
	\begin{center}
		{\Usefont{phv} \Large \bf \title \par}
		\vspace*{40pt}
		\large \em \Usefont{pzc}{ 
			A Project Report \par
			Submitted to the APJ Abdul Kalam Technological University\\
			in partial fulfillment of requirements for the award of degree}\\ [.15\baselineskip] \par
		\Usefont{ppl} {\bfseries  \degree}\\
		in\\
		{\Usefont{ppl} {\bfseries \branch}}\\
		by\\
		\bf {\ Akhil Joseph}(\ KSD21CS012)\\
		\bf {\ Aravind Ramesh}(\ KSD21CS023)\\
		\bf {\ Aswin K}(\ KSD21CS028)\\
		\bf {\ Muhammed Irshad K}(\ KSD21CS072)\\
		\vspace*{40pt}
		\centering
		\begin{figure}[h!]
			\centerline{\includegraphics[scale=0.4]{figures/lbslogo.png}}
		\end{figure}
		
		\vspace{\stretch{0.5}}
		\footnotesize{\bf DEPARTMENT OF COMPUTER SCIENCE AND ENGINEERING} \par
		\bf{LBS COLLEGE OF ENGINEERING KASARAGOD} \par
		\bf{KERALA} \par
		\bf{\month}
	\end{center}		
\end{titlepage}	
 %Unless essential Do not edit this tex file


%****************************Mission and vision***************************
% The font pages. The source tex files are there in the folder
\include{Department Vision and Mission.tex} %Unless essential Do not edit this tex file



%%********************Certificate*******************

% To print name of only the project coordinator 1 in the certificate page
\newenvironment{certificate1}

	\newpage
	\begin{center}	
		%\vspace{1.5cm}
		
		\textbf{DEPT. OF COMPUTER SCIENCE \&  ENGINEERING\\}	
		\textbf{LBS COLLEGE OF ENGINEERING\\}	
		\textbf{KASARAGOD}
		
		\textbf{\acadyear} 
	\end{center}
	
	\begin{center}
		\includegraphics[scale=0.3]{lbslogo.png}	
	\end{center}
	\begin{center}
		\textbf{CERTIFICATE}
	\end{center}
	
	This is to certify that the report entitled \textbf{\title} submitted by \textbf{\ Akhil Joseph}\hspace*{2pt}(\ KSD21CS012),\hspace*{2pt}\textbf{\ Aravind Ramesh}\hspace*{2pt}(\ KSD21CS023),\hspace*{2pt}\textbf{\ Aswin k}\hspace*{2pt}(\ KSD21CS028) \& \textbf{\ Muhammed Irshad k}\hspace*{2pt}(\ KSD21CS072) to the APJ Abdul Kalam Technological University in partial fulfillment of the B.Tech.\ degree in \branch \hspace*{2pt} is a bonafide record of the project work carried out by him under our guidance and supervision. This report in any form has not been submitted to any other University or Institute for any purpose.
	
	
	\begin{singlespace}
		\vspace*{2cm}
		\begin{table}[h!]
			\centering
			\begin{tabular}{p{7cm} p{0.9cm} p{7cm}} 
				\textbf{\guide} && \textbf{\projcordinatorA} \\
				(Project Guide) &&  (Project Coordinator)\\
				\guidedes & & \projcordinatorAdes\\ 
				Dept.of CSE && Dept.of CSE\\ 
				LBS College of Engineering & &LBS College of Engineering\\
				Kasaragod && kasaragod\\
			\end{tabular}
			
		\end{table}
		
		\vspace*{1.3cm}
		
		\begin{center}
			
			%\hline
			\textbf{\hod} \\ 
			\hoddes\\ 
			Dept.of CSE\\ 
			LBS College of Engineering\\
			Kasaragod\\
			
		\end{center}
	\end{singlespace}
	
	\thispagestyle{empty}



 

% To print names of both the project coordinators in the certificate page
%\include{certificate2} %Please uncomment this and comment the previous line

%%***************************************************


\include{declaration} %Unless essential Do not edit this tex file

\pagenumbering{roman} 

%%********************************Abstract***********************
\chapter*{Abstract}%
%\addcontentsline{toc}{chapter}{\numberline{}Abstract}%
\addcontentsline{toc}{chapter}{Abstract}%

 
Agriculture is the backbone of our country. Nowadays so many advanced technologies
in the field of science and technologies are utilizing to improve productivity and quality
of crops generated through agriculture. One of the oldest concepts in electronics field
that incorporate with agriculture field is automation. A single controller that has some
set of sensors and actuators like moisture sensor, temperature sensor, humidity sensor,
PH sensor, Motor Pumps, Draft Fans etc. Most of these automations are working
with respect to the codes programmed on the controller board. After this automated
systems, IoT concept is added in the internet era for monitor parameters and trigger
actuators from anywhere in the world using smartphone application.
In case of large fields, a single controller can’t take data from sensors placed in a wide
area of lands through wires. Even if we increase the data transmission cables length
with a high-quality copper wire there must be some data loss due to copper loss. To
overcome this problem, we need to develop multiple nodes of wireless network that
can communicate each other through some media like internet.\\
Here we are going to develop a system that has so many individual nodes that can
perform specific tasks. For example, a green house has 3 types of crops, each crop
needs different amount of water. Here we implement 3 nodes that can sense the soil
moisture of each crop individually. And there is a 4th node that can control irrigation
network of each node separately through some solenoid valves or motor pumps. Whennode one sense a water deficiency in its soil, it will establish a communication between
there 4th node and tell it to activate irrigation supply to the 1 st crop. When the
water required for this crop reaches it’s limit, 1st node send a request to 4th node
to stop the irrigation supply to 1st crop. Above mentioned is just an example of this
project. We will implement some nodes that can communicate each other to manage

parameters like soil moisture, temperature, humidity etc. Alone with wireless sensor
node networking, machine to machine communication is also implementing here. \\To
reduce the power consumption, we can add low power consumption hardware and
simple encryption algorithms to send data through wireless network. We can also
monitor the sensor data uploaded by these nodes to an IoT server through our custommade android application. Application can also able to control the actuators like pump
and fan from any corner of the world if we have an active internet connection. % 
 
%%***************************************************
% Default Acknowledgement page
\include{acknowledgement}  %Unless essential Do not edit this tex file


%%***************************************************
%%**If you have only one seminar coordinator faculty member
% please comment the above line and uncomment this line

%\include{acknowledgement1}  %Unless essential Do not edit this tex file
%*******************************************************************

\thispagestyle{empty}
\newpage
    %%**********************Table of Contents***********************
\tableofcontents
\listoffigures
\listoftables
\include{symbol} %List of Symbols (Optional) comment if not required.
% symbold may be added in the file symbol.tex



%%********************Body of the report**********
% Arabic numbering is used in the body of the report

\cleardoublepage
\setcounter{page}{1}
\pagenumbering{arabic}

%%********************Chapter 1**********
\chapter{Introduction}
\lipsum[1] % Please comment this line and type in the introduction chapter

%%********************Chapter 2**********
\chapter{Literature Review}
Each chapter is to begin with a brief introduction (in 4 or 5 sentences) about its contents. The contents can then be presented below organised into sections and subsections.

Technical writing is writing or drafting technical communication used in technical and occupational fields\cite{india}, such as computer hardware and software\cite{rpi}, engineering, chemistry, aeronautics, robotics, finance\cite{japan}, medical, consumer electronics, biotechnology, and forestry. Technical writing encompasses the largest sub-field in technical communication. See figure \ref{net2} that shows the autonomous systems in Internet.

    \begin{figure}[h!]
    	\centering
    	\includegraphics[width=0.9\linewidth]{ospf}
    	\caption{Autonomous System Hierarchy}
    	\label{net2}
    \end{figure}

\section{section1}
\lipsum[2] % Please comment this line and type in the content


\subsection{title 2}
\lipsum[3] % Please comment this line and type in the content

\noindent The system is described by the equation \ref{sys_eq1} below. Here y is the ordinate and x is the abscissa , m is the slope and c a constant.

\begin{equation} \label{sys_eq1}
y = mx + c
\end{equation}
\noindent Page centered and unnumbered multiple equations. The * symbol supresses equation numbering.
% Page centered and unnumbered equations
\begin{align*}
2x - 5y &=  8 \\ 
3x + 9y &=  -12
\end{align*}

\noindent Side by side figures can be created using this environment. See fig \ref{wave} below.
\begin{figure}[h!]
	\centering
	\begin{subfigure}[b]{0.4\textwidth}
		\includegraphics[width=\textwidth]{sinewave}
		\caption{Sine Wave}
		\label{fig:1}
	\end{subfigure}
	\hspace{20mm}
	\begin{subfigure}[b]{0.4\textwidth}
		\includegraphics[width=\linewidth]{cosine}
		\caption{Cosine Wave}
		\label{fig:2}
	\end{subfigure}
\caption{The Sine and Cosine waves}
\label{wave}
\end{figure}

%%********************Chapter 3**********
\chapter{System Development}
Each chapter is to begin with a brief introduction (in 4 or 5 sentences) about its contents. The contents can then be presented below organised into sections and subsections.

\section{section1}
\lipsum[2] % Please comment this line and type in the content


\subsection{title 2}
\lipsum[3] % Please comment this line and type in the content

%%********************Chapter 4**********
\chapter{Results and Discussion}
Each chapter is to begin with a brief introduction (in 4 or 5 sentences) about its contents. The contents can then be presented below organised into sections and subsections.

\section{section1}
\lipsum[2] % Please comment this line and type in the content


\subsection{title 2}
\lipsum[3] % Please comment this line and type in the content
\begin{table}[h!]
	\centering
	\caption{test table}
	\vspace*{5pt}
	\begin{tabular}{|c|c|c|}
		\hline
		Sl. No & Item 1 & Itm 2 \\ \hline
		1      & 37     & 45    \\ \hline
		2      & 42     & 23    \\ \hline
		3      & 47     & 1     \\ \hline
		4      & 52     & -21   \\ \hline
		5      & 57     & -43   \\ \hline
		6      & 62     & -65   \\ \hline
		7      & 67     & -87   \\ \hline
		8      & 72     & -109  \\ \hline
		9      & 77     & -131  \\ \hline
		10     & 82     & -153  \\ \hline
	\end{tabular}
\end{table}

%%********************Chapter 5**********
\chapter{Conclusion}
Each chapter is to begin with a brief introduction (in 4 or 5 sentences) about its contents. The contents can then be presented below organised into sections and subsections.

\lipsum[2] 

%%********************References**********
%%****This template uses IEEE bibliography style

 \begin{thebibliography}{99}
	\bibitem{ref1} Bashar, Abul. (2019). AGRICULTURAL MACHINE AUTOMATION USING IOT THROUGH ANDROID. Journal of Electrical Engineering and Automation. 1. 83-92. 10.36548/jeea.2019.2.003. 
	
	
	\bibitem{ref2}
	Math, Rajinderkumar & Dharwadkar, Nagaraj. (2017). A wireless sensor network based low cost and energy efficient frame work for precision agriculture. 
 \bibitem{ref3}
	Raj, A., Srinivasan, B., Abhishek, D., Jeyavanth, S., & Kannan, V.A. (2016). IoT based Agro Automation System using Machine Learning Algorithms.
 
 \bibitem{ref4}
	Olly Roy Chowdhury , Sarna Majumder, 2021, Prospect of IoT Based Smart Agriculture in Bangladesh-A Review, INTERNATIONAL JOURNAL OF ENGINEERING RESEARCH & TECHNOLOGY (IJERT) Volume 10, Issue 11 (November 2021),
 \bibitem{ref5}
	García L, Parra L, Jimenez JM, Lloret J, Lorenz P. IoT-Based Smart Irrigation Systems: An Overview on the Recent Trends on Sensors and IoT Systems for Irrigation in Precision Agriculture. Sensors (Basel). 2020 Feb 14;20(4):1042. doi: 10.3390/s20041042. PMID: 32075172; PMCID: PMC7070544.
 

	\bibitem{ref6} Bhoi, A., Nayak, R. P., Bhoi, S. K., Sethi, S., Panda, S. K., Sahoo, K. S., & Nayyar, A. (2021). IoT-IIRS: Internet of Things based intelligent-irrigation recommendation system using machine learning approach for efficient water usage. PeerJ Computer Science, 7, e578. 	
\end{thebibliography}

\end{document}