\chapter*{Abstract}%
%\addcontentsline{toc}{chapter}{\numberline{}Abstract}%
\addcontentsline{toc}{chapter}{Abstract}%

 
Agriculture is the backbone of our country. Nowadays so many advanced technologies
in the field of science and technologies are utilizing to improve productivity and quality
of crops generated through agriculture. One of the oldest concepts in electronics field
that incorporate with agriculture field is automation. A single controller that has some
set of sensors and actuators like moisture sensor, temperature sensor, humidity sensor,
PH sensor, Motor Pumps, Draft Fans etc. Most of these automations are working
with respect to the codes programmed on the controller board. After this automated
systems, IoT concept is added in the internet era for monitor parameters and trigger
actuators from anywhere in the world using smartphone application.
In case of large fields, a single controller can’t take data from sensors placed in a wide
area of lands through wires. Even if we increase the data transmission cables length
with a high-quality copper wire there must be some data loss due to copper loss. To
overcome this problem, we need to develop multiple nodes of wireless network that
can communicate each other through some media like internet.\\
Here we are going to develop a system that has so many individual nodes that can
perform specific tasks. For example, a green house has 3 types of crops, each crop
needs different amount of water. Here we implement 3 nodes that can sense the soil
moisture of each crop individually. And there is a 4th node that can control irrigation
network of each node separately through some solenoid valves or motor pumps. Whennode one sense a water deficiency in its soil, it will establish a communication between
there 4th node and tell it to activate irrigation supply to the 1 st crop. When the
water required for this crop reaches it’s limit, 1st node send a request to 4th node
to stop the irrigation supply to 1st crop. Above mentioned is just an example of this
project. We will implement some nodes that can communicate each other to manage

parameters like soil moisture, temperature, humidity etc. Alone with wireless sensor
node networking, machine to machine communication is also implementing here. \\To
reduce the power consumption, we can add low power consumption hardware and
simple encryption algorithms to send data through wireless network. We can also
monitor the sensor data uploaded by these nodes to an IoT server through our custommade android application. Application can also able to control the actuators like pump
and fan from any corner of the world if we have an active internet connection.